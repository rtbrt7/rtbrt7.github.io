% Awesome Source CV LaTeX Template
%
% This template has been downloaded from:
% https://github.com/darwiin/awesome-neue-latex-cv
%
% Author:
% Christophe Roger
%
% Template license:
% CC BY-SA 4.0 (https://creativecommons.org/licenses/by-sa/4.0/)

%Section: Project
\sectionTitle{Working Paper}{\faLaptop}

\begin{projects}
	\project
	{Enforcement of Non-Tariff Measures: Does it matter?}{2020}
	{\website{https://www.iimb.ac.in/node/7662}{IIMB Working Paper 611} (with Bernardo Blum and Kunal Dasgupta)}
	{We use data and a textual analysis algorithm to uncover that at least 28\% of the Singaporean imports are affected by per shipment compliance costs created by non-tariff trade barriers which in turn, have a large direct welfare effect on the economy, when the compliance creates a fixed cost.}
	{International Trade, Per-shipment cost, Non-Tariff Barriers, Enforcement, Natural Language Processing, Theoretical Model, Quantitative Exercise}
				
	\project
	{Globalization and Workforce Composition in Indian Formal Manufacturing:  New evidence on product market competition channel}{2020}
	{\website{https://www.researchgate.net/publication/346107946_Globalization_and_Workforce_Composition_in_Indian_Formal_Manufacturing_New_evidence_on_product_market_competition_channel}{IGIDR Working Paper 2020-036} (with K.V. Ramaswamy) \hfill {\color{accentcolor}{(Revise and Resubmit)}}}
	{We investigate the indirect impact of globalization on job quality (workforce composition) through changes in product market structure for the Indian formal manufacturing industries. We found that the firms responded to greater product market competition due to globalization by hiring relatively more contract workers. It enabled them to achieve labor market flexibility.}
	{International Trade, Industrial Organisation, Segmented Labour Market, Formal Manufacturing Sector, Market Concentration, Dynamic Panel Model}

	\project
	{Disaggregated Indian Industrial Cycles: A spectral analysis}{2020}
	{\website{https://www.researchgate.net/publication/344764797_Disaggregated_Indian_Industrial_Cycles_A_spectral_analysis}{IGIDR Working Paper 2020-033} (with Ashima Goyal) \hfill {\color{accentcolor}{(under review)}}}
	{First, we study the characteristics of asymmetric phase shifts and co-movement of industrial production cycles. Then we check the spectral causality of these cycles with various policy variables, which enabled us to explore the non-linear structure of the economy by computing Granger Causality for each frequency separately.}
	{Industrial Cycles, Business Cycle Dating, Coherence, Spectral Causality, Co-movement, Lead/Lag Industries, Macroeconomic Stabilization}

\end{projects}